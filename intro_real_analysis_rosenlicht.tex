\documentclass{article}
\usepackage[utf8]{inputenc}

\title{Intro Real Analysis}
\author{Rosenlicht}
\date{April 2022}

\usepackage{amsmath}
\usepackage{amsthm}
\usepackage{amssymb}
\usepackage{bm}

\newtheorem{definition}{Definition}[section]
\newtheorem{theorem}{Theorem}[section]
\newtheorem{corollary}{Corollary}[theorem]

\begin{document}
	
	\maketitle
	\tableofcontents
	
	\section{Notions from Set Theory}
		\subsection{Sets and Elements. Subsets}
		
		\subsection{Operations on Sets}
		
		\subsection{Functions}
		
		\subsection{Finite and Infinite Sets}
	
	\section{The Real Number System}
		\subsection{The Field Properties}
		
		\subsection{Order}
		
		\subsection{The Least Upper Bound Property}
		
		\subsection{The Existence of Square Roots}

	
	\section{Metric Spaces}
		\subsection{Definition of Metric Spaces. Examples}
			
			\begin{definition}[Metric Space]
				A Metric Space is an ordered pair $(E, d)$ where $E$ is some set along with a metric function $d \colon E \times E \to \mathbb{R}$
				 such that:
				 \begin{enumerate}
				 	\item \textbf{Identity of Indiscernibles:} $d(p,q) = 0 \Leftrightarrow p=q, \hspace{2 mm} \forall p,q \in E$
				 	\item \textbf{Symmetry:} $d(p,q) = d(q,p), \hspace{2 mm} \forall p,q \in E$
				 	\item \textbf{Subadditivity:} $d(p,q) \leq d(p,r) + d(r,q), \hspace{2 mm} \forall p,q,r \in E$
				 \end{enumerate}
			\end{definition}
		
			\begin{theorem}
				Metric functions are non-negative
			\end{theorem}
			\begin{proof}
				Let $d$ be a a metric function with points $p,q \in E$. So,
				\begin{align*}
					0 &= d(p,p) \tag{Identity of Indiscernibles}\\
					  &= \dfrac{d(p,p)}{2}\\
					  &\leq \dfrac{d(p,q) + d(q,p)}{2} \tag{Subadditivity}\\
					  &= \dfrac{d(p,q) + d(p,q)}{2} \tag{Symmetry}\\
					  &= 2 \cdot \dfrac{d(p,q)}{2}\\
					  &= d(p,q) &&\qedhere
				\end{align*}
			\end{proof}

			\begin{theorem}
				$d(p_1,p_n) \leq \sum\limits_{k=1}^{n-1} d(p_k,p_{k+1}), \hspace{2 mm} \forall p_1,\ldots,p_n \in E$
			\end{theorem}
			\begin{proof}
				We show the above statement via Induction. Note the case for $n=2$ is trivial so we omit it: \\
				Base case $(n=3)$
				$$d(p_1,p_3) \leq d(p_1,p_2) + d(p_2,p_3) = \sum\limits_{k=1}^{3-1} d(p_k,p_{k+1})$$
				Inductive Step $\left(n \longrightarrow n+1\right)$
				$$d(p_1,p_{n+1}) \leq d(p_1,p_n) + d(p_n,p_{n+1}) 
								 \leq \sum\limits_{k=1}^{n-1} d(p_k,p_{k+1}) + d(p_n,p_{n+1})
								 = \sum\limits_{k=1}^{(n+1)-1} d(p_k,p_{k+1})$$
			\end{proof}

			\begin{theorem}
				$|d(p,r) - d(q,r)| \leq d(p,q), \hspace{2 mm} \forall p,q,r \in E$
			\end{theorem}
			\begin{proof}
				Let $p,q,r \in E$. Then:
				\begin{align*}
					d(q,r) &\leq d(q,p)+d(p,r) \tag{Subadditivity}\\
					d(q,r) &\leq d(p,q)+d(p,r) \tag{Symmetry}\\
					-d(q,r) &\geq -d(p,q)-d(p,r)\\
					d(p,r)-d(q,r) &\geq -d(p,q)
				\end{align*}
				We also have:
				\begin{align*}
					d(p,r) &\leq d(p,q)+d(q,r) \tag{Subadditivity}\\
					d(p,r)-d(q,r) &\leq d(p,q)
				\end{align*}	
				Which gives us:
				$$-d(p,q) \leq d(p,r)-d(q,r) \leq d(p,q)$$
				$$|d(p,r) - d(q,r)| \leq d(p,q)$$

			\end{proof}

			\begin{definition}[Dot Product]
				The Dot Product of vectors $\bm{x}, \bm{y} \in \mathbf{R}^n$ is $$\bm{x} \cdot \bm{y} = \sum\limits_{i=1}^n x_i y_i$$
			\end{definition}

			\begin{definition}[Euclidean Norm]
				The Euclidean norm of $\bm{x} \in \mathbf{R}^n$ is $$||\bm{x}|| = \sqrt{\sum\limits_{i=1}^n x_i^2} = \sqrt{\bm{x} \cdot \bm{x}}$$
			\end{definition}

			\begin{theorem}[Cauchy Schwarz Inequality]
			\label{cauchy-schwarz}
				$|\bm{x}\cdot\bm{y}| \leq ||\bm{x}||\cdot||\bm{y}||, \hspace{2 mm} \forall \bm{x}, \bm{y} \in \mathbb{R}^n$
			\begin{proof}
				Let $\bm{x}, \bm{y} \in \mathbb{R}^n$. Then,
				\begin{align*}
					||t \cdot \bm{y} + \bm{x}||^2 &\geq 0, \hspace{2 mm} \forall t \in \mathbb{R}^n\\
					\sum\limits_{i=1}^n {(t \cdot y_i + x_i)}^2 &\geq 0\\
					\sum\limits_{i=1}^n (t^2 \cdot y_i^2 + 2x_i y_i t+x_i^2) &\geq 0\\
					t^2 \sum\limits_{i=1}^n y_i^2 + 2t \sum\limits_{i=1}^n x_i y_i + \sum\limits_{i=1}^n x_i^2 &\geq 0\\
					t^2 ||\bm{y}||^2 + 2t (\bm{x}\cdot\bm{y}) + ||\bm{x}||^2 &\geq 0\\
				\end{align*}
				The above is a quadratic in $t$ that is non-negative. Therefore it must have a non-positive discriminant.
				\begin{align*}
					(2(\bm{x}\cdot\bm{y}))^2 - 4||\bm{y}||^2 ||\bm{x}||^2 &\leq 0\\
					4(\bm{x}\cdot\bm{y})^2 - 4||\bm{y}||^2 ||\bm{x}||^2 &\leq 0\\
					(\bm{x}\cdot\bm{y})^2 - ||\bm{y}||^2 ||\bm{x}||^2 &\leq 0\\
					(\bm{x}\cdot\bm{y})^2 - (||\bm{x}|| \cdot ||\bm{y}||)^2 &\leq 0\\
					(\bm{x}\cdot\bm{y})^2 &\leq (||\bm{x}|| \cdot ||\bm{y}||)^2\\
					|\bm{x}\cdot\bm{y}| &\leq ||\bm{x}|| \cdot ||\bm{y}|| &&\qedhere
				\end{align*}
			\end{proof}
			\end{theorem}

			\begin{corollary}
				The Euclidean Norm is Subadditive.
			\end{corollary}
			\begin{proof}
				Let $\bm{x}, \bm{y} \in \mathbf{R}^n$. Then,
				\begin{align*}
					||\bm{x} + \bm{y}||^2 &= \sum\limits_{i=1}^n (x_i + y_i)^2\\
										  &= \sum\limits_{i=1}^n (x_i^2 + 2x_i y_i + y_i^2)\\
										  &= \sum\limits_{i=1}^n x_i^2 + 2 \sum\limits_{i=1}^n x_i y_i + \sum\limits_{i=1}^n y_i^2\\
										  &= ||\bm{x}||^2 + 2 (\bm{x} \cdot \bm{y}) + ||\bm{y}||^2\\
										  &\leq ||\bm{x}||^2 + 2 ||\bm{x}|| \cdot ||\bm{y}| + ||\bm{y}||^2 \tag{Cauchy Schwarz Inequality}\\
										  &= (||\bm{x}|| + ||\bm{y}||)^2\\
					  ||\bm{x} + \bm{y}|| &\leq ||\bm{x}|| + ||\bm{y}|| &&\qedhere
				\end{align*}
				
			\end{proof}

			\begin{theorem}
			\label{euclidean metric space}
				$\mathbb{R}^n$ along with $d(\bm{p},\bm{q}) = ||\bm{p}-\bm{q}||$ forms a Metric Space.
			\end{theorem}
			\begin{proof}
				Let $\bm{p}, \bm{q} \in \mathbb{R}^n$ 
				\item \textbf{Identity of Indiscernibles:} $d(\bm{p},\bm{p}) = ||\bm{p}-\bm{p}|| = ||\bm{0}|| = 0$
				\item \textbf{Symmetry:} $d(\bm{p},\bm{q}) = ||\bm{p}-\bm{q}|| = ||\bm{q}-\bm{p}|| = d(\bm{q},\bm{p})$
				\item \textbf{Subadditivity:}
				\begin{align*}
					d(\bm{p},\bm{q}) &= ||\bm{p}-\bm{q}||\\
									 &= ||(\bm{p}-\bm{r}) + (\bm{r}-\bm{q})||\\
									 &\leq ||(\bm{p}-\bm{r})|| + ||(\bm{r}-\bm{q})|| \tag{Subadditivity of $||\cdot||$}\\
									 &= d(\bm{p},\bm{r}) + d(\bm{r},\bm{q}) &&\qedhere
				\end{align*}
			\end{proof}

			\begin{corollary}
				$\mathbb{R}$ along with $d(p, q) = |p - q|$ forms a Metric Space.
			\end{corollary}
			\begin{proof}
				Let n = 1. Then by Theorem \ref{euclidean metric space} we obtain a metric space
			\end{proof}

			\begin{theorem}[Taxicab metric space]
				\label{taxicab metric space}
				Let E be some set along with 
				\begin{equation*}
					d(p, q)=\begin{cases}
							  1 \quad & p \neq q \\
							  0 \quad & p = q \\
						 \end{cases}
					\end{equation*}
				Then $(E, d)$ forms a Metric Space.
				\end{theorem}
				\begin{proof}
					Let $p, q, r \in E$ 
					\item \textbf{Identity of Indiscernibles:} $d(p, p) = 0$
					\item \textbf{Symmetry:} \\
						Let $p \neq q$
						$$d(p, q) = 1 = d(q, p)$$
						Let $p=q$
						$$d(p, q) = 0 = d(q, p)$$
					\item \textbf{Subadditivity:}
						\begin{equation*}
							d(p, r) + d(r, q)=\begin{cases}
									2 \quad & p \neq r \neq q \\
									0 \quad & p = q = r \\
									1 \quad &else
								\end{cases}
						\end{equation*}
						Therefore we have $d(p, q) \leq d(p, r) + d(r, q)$

				\end{proof}
		\subsection{Open and Closed Sets}
		
		\subsection{Convergent Sequences}
		
		\subsection{Completeness}
		
		\subsection{Compactness}
		
		\subsection{Connectedness}
	
	\section{Continuous Functions}
		\subsection{Definition of Continuity. Examples}
		
		\subsection{Continuity and Limits}
		
		\subsection{The Continuity of Rational Operations. Functions with values in $E^n$}
		
		\subsection{Continuous Functions on a Compact Metric Space}
		
		\subsection{Continuous Functions on a Connected Metric Space}
		
		\subsection{Sequences of Functions}
	
	\section{Differentiation}
		\subsection{Definition of the Derivative}
		
		\subsection{Rules of Differentiation}		
		
		\subsection{The Mean Value Theorem}
		
		\subsection{Taylor's Theorem}	
	
	\section{Riemann Integration}
		\subsection{Definition and Examples}
		
		\subsection{Linearity and Order Properties of the Integral}
		
		\subsection{Existence of the Integral}
		
		\subsection{The Fundamental Theorem of Calculus}
		
		\subsection{The Logarithmic and Exponential Functions}
		
		\subsection{Definition of Continuity. Examples}
	
	
	\section{Interchange of Limit Operations}
		\subsection{Integration and Differentiation of Sequences of Functions}
		
		\subsection{Infinite Series}
		
		\subsection{Power Series}
		
		\subsection{The Trigonometric Functions}
		
		\subsection{Differentiation under the Integral Sign}
		
		
	\section{The Method of Successive Approximations}
		\subsection{The Fixed Point Theorem}
		
		\subsection{The Simplest Case of the Implicit Function Theorem}
		
		\subsection{Existence and Uniqueness Theorems for Ordinary Differential Equations}
	
	
	\section{Partial Differentiation}
		\subsection{Definitions and Basic Properties}
		
		\subsection{Higher Derivatives}
		
		\subsection{The Implicit Function Theorem}
	
	
	\section{Multiple Integrals}
		\subsection{Riemann Integration on Closed Intervals of $E^n$. Examples and Basic Properties}
		
		\subsection{Existence of the Integral. Integration on Arbitrary Subsets of $E^n$. Volume}
		
		\subsection{Iterated Integrals}
		
		\subsection{Change of Variable}

	
\end{document}
