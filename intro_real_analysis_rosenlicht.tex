\documentclass{article}
\usepackage[utf8]{inputenc}

\title{Intro Real Analysis}
\author{Rosenlicht}
\date{April 2022}

\usepackage{amsmath}
\usepackage{amsthm}
\usepackage{amssymb}
\usepackage{bm}

\newtheorem{definition}{Definition}[section]
\newtheorem{theorem}{Theorem}[section]
\newtheorem{corollary}{Corollary}[theorem]

\begin{document}
	
	\maketitle
	\tableofcontents
	
	\section{Notions from Set Theory}
		\subsection{Sets and Elements. Subsets}
		
		\subsection{Operations on Sets}
		
		\subsection{Functions}
		
		\subsection{Finite and Infinite Sets}
	
	\section{The Real Number System}
		\subsection{The Field Properties}
		
			\begin{definition}[Group]
			\label{group}
				A Group is a an ordered pair $(G, *)$ where $G$ is some non-empty set along with a closed binary operator $*\colon G \times G \to G$ s.t.:
				\begin{enumerate}
					\item \textbf{Associative:} $\forall x, y, z \in G: (x * y) * z = x * (y * z)$.
					\item \textbf{Identity Element:} $\forall x \in G, \exists! i_G \in G: x * i_G = i_G * x = x$.
					\item \textbf{Inverse Element:} $\forall x \in G, \exists! x^{-1} \in G: x * x^{-1} = x^{-1} * x = i_G$.
				\end{enumerate}
			\end{definition}

			\begin{definition}[Abelian Group]
			\label{abelian group}
				An Abelian Group is a Commutative Group:
				$$\forall x, y \in G: x * y = y * x$$
			\end{definition}

			\begin{definition}[Field]
			\label{field}
				A Field is an ordered triple $(\mathbb{F}, *, \circ)$ s.t.:
				\begin{enumerate}
					\item $(\mathbb{F}, *)$ and $(\mathbb{F}, \circ)$ form Abelian Groups with $0_\mathbb{F} = i_{(\mathbb{F}, *)}$ and $1_\mathbb{F} = i_{(\mathbb{F}, \circ)}$.
					\item \textbf{Distributive Property:} $\forall x, y, z \in \mathbb{F}, x * (y \circ z) = (x * y) \circ (x * z)$.
				\end{enumerate}
			\end{definition}
			
			\begin{definition}[Real Numbers]
			\label{real numbers}
				$(\mathbb{R}, +, \cdot)$ forms a Field where:
				\begin{enumerate}
					\item \textbf{Addition:} $+\colon \mathbb{R} \times \mathbb{R} \to \mathbb{R}$, $i_\mathbb{R} = 0$. 
					\item \textbf{Multiplication:} $\cdot \colon \mathbb{R} \times \mathbb{R} \to \mathbb{R}$, $i_\mathbb{R} = 1$, where 0 has no inverse element.
				\end{enumerate}
			\end{definition}

			Some other common Fields are $(\mathbb{Q}, +, \cdot)$ and $(\mathbb{C}, +, \cdot)$. 
			So far, these have all been infinite Fields, but one can have finite Fields as well.
			The smallest possible finite Field is $(\{0_\mathbb{F}, 1_\mathbb{F} \}, *, \circ)$. Below are some immediate consequences of our algebraic structures:
			\begin{enumerate}
				\item For any Associative Closed Binary Operation, parantheses can be omitted:
				$\cdot \colon S \times S \to S$ associative $\Rightarrow x \cdot y \cdot z = (x \cdot y) \cdot z = x \cdot (y \cdot z), \forall x, y, z \in S$.
				
				\item For any Abelian Group, the order of elements is immaterial: \\
				$(G, \cdot)$  Abelian Group $\Rightarrow x \cdot y \cdot z = y \cdot x \cdot z = y \cdot z \cdot x = z \cdot y \cdot x, \forall x, y, z \in G$.
				
				\item For any Group, the equation $x \cdot y = z$ has a unique solution in $x$: \\
				$(G, \cdot)$ Group $\Rightarrow \forall y, z \in G, \exists! x \in G: x \cdot y = z$.
				\begin{proof}
					Let $y, z \in (G, \cdot)$.\\
					$\Rightarrow y^{-1} \in G$.\\
					$\Rightarrow z \cdot y^{-1} \in G$.\\
					$\Rightarrow (z \cdot y^{-1}) \cdot y = z \cdot (y^{-1} \cdot y)$.\\
					$\Rightarrow (z \cdot y^{-1}) \cdot y = z \cdot i_G$.\\
					$\Rightarrow (z \cdot y^{-1}) \cdot y = z$.\\
					$\Rightarrow x = z \cdot y^{-1}$ is a solution to $x \cdot y = z$.\\
					Let $w$ be some other solution to $x \cdot y = z$.\\
					$\Rightarrow w \cdot y = z$.\\
					$\Rightarrow (w \cdot y) \cdot y^{-1} = z \cdot y^{-1}$.\\
					$\Rightarrow w \cdot (y \cdot y^{-1}) = z \cdot y^{-1}$.\\
					$\Rightarrow w \cdot i_G = z \cdot y^{-1}$.\\
					$\Rightarrow w = z \cdot y^{-1}$.\\
					$\Rightarrow w = x$.\\
					Therefore $ x = z \cdot y^{-1}$ is a unique solution to $x \cdot y = z$.\\		
				\end{proof}

				\item For any Group, the identity element is unique:
				$$\forall x \in G, \exists ! i_G \in G: i_G \cdot x = x \cdot i_G = x$$
				\begin{proof}
					Let $(G, \cdot)$ be a Group with identity element $i_G$.\\
					Let $y, z \in G$ s.t. $y = z$.\\
					$\Rightarrow i_G = y \cdot y^{-1} = z \cdot y^{-1}$.\\
					$\Rightarrow x = i_G$ is a unique solution to $x \cdot y = z$.\\
				\end{proof}

				\item For any Group, inverse elements are unique:
				$$\forall x \in G, \exists ! x^{-1} \in G: x \cdot x^{-1} = x^{-1} \cdot x = i_G$$
				\begin{proof}
					Let $(G, \cdot)$ be a Group with identity element $i_G$.\\
					Let $y, z \in (G, \cdot)$ s.t. $z = i_G$.\\
					$\Rightarrow y^{-1} = i_G \cdot y^{-1} = z \cdot y^{-1}$.\\
					$\Rightarrow x = y^{-1}$ is a unique solution to $x \cdot y = z$.
				\end{proof}

				\item For any Field, any element multiplied by the Additive identity element yields the Additive identity element.
				\begin{proof}
					Let $x \in (\mathbb{F}, +, \cdot)$.\\
					$\Rightarrow x \cdot 0_{\mathbb{F}} = x \cdot i_{(\mathbb{F}, +)} = x \cdot (i_{(\mathbb{F}, +)} + i_{(\mathbb{F}, +)}) = (x \cdot i_{(\mathbb{F}, +)}) + (x \cdot i_{(\mathbb{F}, +)})$.\\
					Let $y = x \cdot i_{(\mathbb{F}, +)}$.\\
					$\Rightarrow y = y + y$.\\
					$\Rightarrow y + y^{-1} = (y + y) + y^{-1}$.\\
					$\Rightarrow y + y^{-1} = y + (y + y^{-1})$.\\
					$\Rightarrow i_{(\mathbb{F}, +)} = y + i_{(\mathbb{F}, +)}$.\\
					$\Rightarrow i_{(\mathbb{F}, +)} = y$.\\
					$\Rightarrow 0_{\mathbb{F}} = y$.\\
					$\Rightarrow 0_{\mathbb{F}} = x \cdot 0_{\mathbb{F}}$.
					$\Rightarrow 0_{\mathbb{F}} \cdot x = i_{(\mathbb{F}, +)} \cdot x = (i_{(\mathbb{F}, +)} + i_{(\mathbb{F}, +)}) \cdot x = (i_{(\mathbb{F}, +)} \cdot x) + (i_{(\mathbb{F}, +)} \cdot x)$.\\
					Let $z = i_{(\mathbb{F}, +)} \cdot x$.\\
					$\Rightarrow z = z + z$.\\
					$\Rightarrow z + z^{-1} = (z + z) + z^{-1}$.\\
					$\Rightarrow z + z^{-1} = z + (z + z^{-1})$.\\
					$\Rightarrow i_{(\mathbb{F}, +)} = z + i_{(\mathbb{F}, +)}$.\\
					$\Rightarrow i_{(\mathbb{F}, +)} = z$.\\
					$\Rightarrow 0_{\mathbb{F}} = z$.\\
					$\Rightarrow 0_{\mathbb{F}} = 0_{\mathbb{F}} \cdot x$.\\
					$\Rightarrow x \cdot 0_{\mathbb{F}} = 0_{\mathbb{F}} \cdot x = 0_{\mathbb{F}}$.
				\end{proof}
				
				\item For any Field, if the product of 2 elements is the Zero element, then atleast 1 factor must be the Zero element:
				\begin{proof}
					Let $x, y \in (\mathbb{F}, +, \cdot)$ s.t. $x, y \neq 0_{\mathbb{F}}$.\\
					Case 1: $x = 0_{\mathbb{F}} \Rightarrow x \cdot y = 0_{\mathbb{F}} \cdot y = 0_{\mathbb{F}}$.\\
					Case 2: $y = 0_{\mathbb{F}} \Rightarrow x \cdot y = x \cdot 0_{\mathbb{F}} = 0_{\mathbb{F}}$.
				\end{proof}
				\item For any Group, the inverse of the inverse element is the element itself.
				\begin{proof}
					Let $x \in (G, \cdot)$.\\
					$\Rightarrow \exists ! x^{-1} \in G: x \cdot x^{-1} = i_G$ \\
					$\Rightarrow \exists ! ({x^{-1}})^{-1} \in G: ({x^{-1}})^{-1} \cdot x^{-1} = i_G$.\\
					$\Rightarrow$ Both $({x^{-1}})^{-1}$ and $x$ are solutions since they are both the solution to same equation:
					$$x \cdot y = z: y = x^{-1}, z = i_G$$
				\end{proof}

				\item For any Field, we can define negative numbers as $-x = -1_\mathbb{F} \cdot x = x \cdot -1_\mathbb{F}$, where $-1_\mathbb{F}$ is the inverse of $1_\mathbb{F}$ under $+$.
				\begin{proof}
					Let $x \in (\mathbb{F}, +, \cdot)$.\\
					$\Rightarrow \exists !x^{-1} \in (\mathbb{F}, +): x^{-1} + x = 0_\mathbb{F}$.\\
					Define $x^{-1} = -x$.\\
					$\Rightarrow -x + x = 0_\mathbb{F}$.\\
					$\Rightarrow -1_\mathbb{F} \cdot x + x = -1_\mathbb{F} \cdot x + 1_\mathbb{F} \cdot x$.\\
					$\Rightarrow -1_\mathbb{F} \cdot x + x = (-1_\mathbb{F} \cdot + 1_\mathbb{F}) \cdot x$.\\
					$\Rightarrow -1_\mathbb{F} \cdot x + x = 0_\mathbb{F} \cdot x$.\\
					$\Rightarrow -1_\mathbb{F} \cdot x + x = 0_\mathbb{F}$.\\
					Also $x \cdot -1_\mathbb{F} + x = x \cdot -1_\mathbb{F} + x \cdot 1_\mathbb{F}$.\\
					$\Rightarrow x \cdot -1_\mathbb{F} + x = x \cdot (-1_\mathbb{F} \cdot + 1_\mathbb{F})$.\\
					$\Rightarrow x \cdot -1_\mathbb{F} + x = x \cdot 0_\mathbb{F}$.\\
					$\Rightarrow x \cdot -1_\mathbb{F} + x = 0_\mathbb{F}$.\\
					$\Rightarrow -x = -1_\mathbb{F} \cdot x = x \cdot -1_\mathbb{F}$, since all three are solutions to same equation:
					$$x + y = z: y = x, z = 0_\mathbb{F}$$
				\end{proof}

				\item For any Field, multiplying 2 negative numbers yields a positive number.
				\begin{proof}
					Let $x, y \in (\mathbb{F}, +, \cdot)$.\\
					$\Rightarrow -x \cdot -y = (-1_\mathbb{F}) \cdot x \cdot (-1_\mathbb{F}) \cdot y$\\
					$\Rightarrow -x \cdot -y = (-1_\mathbb{F}) \cdot (-x \cdot  y)$\\
					$\Rightarrow -x \cdot -y = -(-x \cdot  y)$\\
					$\Rightarrow -x \cdot -y = x \cdot  y$\\
				\end{proof}

				\item For any Field, the additive inverse of a sum is the sum of the additive inverses.
				\begin{proof}
					Let $x, y \in (\mathbb{F}, +, \cdot)$\\
					$\Rightarrow -(x + y) = (-1_\mathbb{F}) \cdot (x + y)$\\
					$\Rightarrow -(x + y) = (-1_\mathbb{F}) \cdot x + (-1_\mathbb{F}) \cdot y$\\
					$\Rightarrow -(x + y) = (-x) + (-y)$
				\end{proof}

				\item For any Field, the multiplicative inverse of a product of non-zero elements is the product of the multiplicative inverses reversed.
				\begin{proof}
					Let $x, y \in (\mathbb{F}, +, \cdot)$ s.t. $x, y \neq 0_{\mathbb{F}}$.\\
					$\Rightarrow x \cdot y \neq 0_\mathbb{F}$ \\
					$\Rightarrow \exists ! x^{-1}, y^{-1}, (x \cdot y)^{-1} \in (\mathbb{F}, \cdot)$.\\
					$\Rightarrow (x \cdot y)^{-1} \cdot (x \cdot y) = 1_\mathbb{F}$.\\
					$\Rightarrow (x \cdot y)^{-1} \cdot (x \cdot y) \cdot y^{-1} = 1_\mathbb{F} \cdot y^{-1}$.\\
					$\Rightarrow (x \cdot y)^{-1} \cdot x \cdot (y \cdot y^{-1}) = 1_\mathbb{F} \cdot y^{-1}$.\\
					$\Rightarrow (x \cdot y)^{-1} \cdot x \cdot 1_\mathbb{F} = 1_\mathbb{F} \cdot y^{-1}$.\\
					$\Rightarrow (x \cdot y)^{-1} \cdot x = y^{-1}$.\\
					$\Rightarrow (x \cdot y)^{-1} \cdot x \cdot x^{-1} = y^{-1} \cdot x^{-1}$.\\
					$\Rightarrow (x \cdot y)^{-1} \cdot 1_\mathbb{F} = y^{-1} \cdot x^{-1}$.\\
					$\Rightarrow (x \cdot y)^{-1} = y^{-1} \cdot x^{-1}$.\\
				\end{proof}
			\end{enumerate}

		\subsection{Order}
		
		\subsection{The Least Upper Bound Property}
		
		\subsection{The Existence of Square Roots}

	
	\section{Metric Spaces}
		\subsection{Definition of Metric Spaces. Examples}
			\begin{definition}[Metric Space]
				A Metric Space is an ordered pair $(M, d)$ where $M$ is some set along with a metric function $d \colon M^2 \to \mathbb{R}$ s.t. $\forall x, y, z \in (M, d):$
				\begin{enumerate}
					\item \textbf{Symmetry:} $d(x, y) = d(y, x)$
					\item \textbf{Identity of Indiscernibles:} $d(x, y) = 0 \Leftrightarrow x=y$
					\item \textbf{Triangle Inequality:} $d(x, y) \leq d(x, z) + d(z, y)$
				\end{enumerate}
			\end{definition}
		
			\begin{theorem}
				Metric functions are non-negative.
			\end{theorem}
			\begin{proof}
				Let $x, y \in (M, d)$\\
				$\Rightarrow d(x, y) = \dfrac{2 \cdot d(x, y)}{2}$\\
				$\Rightarrow d(x, y) = \dfrac{d(x, y) + d(x, y)}{2}$\\
				$\Rightarrow d(x, y) = \dfrac{d(x, y) + d(y, x)}{2}$\\
				$\Rightarrow d(x, y) \geq \dfrac{d(x, x)}{2}$\\
				$\Rightarrow d(x, y) \geq 0$\\
			\end{proof}

			\begin{theorem}[General Triangle Inequality]
			\label{general triangle inequality}
				For any sequence of points in a metric space:
				$$d(x_1,x_n) \leq \sum\limits_{k=1}^{n-1} d(x_k,x_{k+1})$$
			\end{theorem}
			\begin{proof}
				Let $x_1, \ldots, x_n \in (M, d)$. By induction:
				\begin{enumerate}
					\item Base Case ($n = 3$):\\
						$\Rightarrow d(x_1, x_3) \leq d(x_1, x_2) + d(x_2, x_3)$
					\item Inductive Hypothesis:\\
						$d(x_1,x_n) \leq \sum\limits_{k=1}^{n-1} d(x_k,x_{k+1})$
					\item Inductive Step:\\
						$\Rightarrow d(x_1,x_n) + d(x_n,x_{n+1}) \leq \sum\limits_{k=1}^{n-1}  d(x_k,x_{k+1}) + d(x_n,x_{n+1})$\\
						$\Rightarrow d(x_1,x_n) + d(x_n,x_{n+1}) \leq \sum\limits_{k=1}^n  d(x_k,x_{k+1})$\\
						$\Rightarrow d(x_1,x_{n+1}) \leq \sum\limits_{k=1}^{n} d(x_k,x_{k+1})$
					\end{enumerate}
			\end{proof}

			\begin{theorem}[Reverse Triangle Inequality]
			\label{reverse-triangle-inequality}
			$$|d(x, z) - d(z, y)| \leq d(x, y), \hspace{2 mm} \forall x, y, z \in (M, d)$$
			\end{theorem}
			\begin{proof}
				Let $x, y, z \in (M, d)$.\\
				$\Rightarrow d(y, z) \leq d(y, x) + d(x, z)$ and $d(x, z) \leq d(x, y) + d(y, z)$.\\
				$\Rightarrow d(z, y) \leq d(x, y) + d(x, z)$ and $d(x, z) \leq d(x, y) + d(z, y)$.\\
				$\Rightarrow -d(x, y) \leq d(x, z) - d(z, y)$ and $d(x, z) - d(z, y) \leq d(x, y)$.\\
				$\Rightarrow -d(x, y) \leq d(x, z) - d(z, y) \leq d(x, y)$.\\
				$\Rightarrow |d(x, z) - d(z, y)| \leq d(x, y)$.
			\end{proof}

			\begin{definition}[Dot Product]
				The Dot Product of vectors $\bm{x}, \bm{y} \in \mathbf{R}^n$ is $$\bm{x} \cdot \bm{y} = \sum\limits_{i=1}^n x_i y_i$$
			\end{definition}

			\begin{definition}[Euclidean Norm]
				The Euclidean norm of $\bm{x} \in \mathbf{R}^n$ is $$||\bm{x}|| = \sqrt{\sum\limits_{i=1}^n x_i^2} = \sqrt{\bm{x} \cdot \bm{x}}$$
			\end{definition}

			\begin{theorem}[Cauchy Schwarz Inequality]
			\label{cauchy-schwarz}
				$$|\bm{x}\cdot\bm{y}| \leq ||\bm{x}||\cdot||\bm{y}||, \hspace{2 mm} \forall \bm{x}, \bm{y} \in \mathbb{R}^n$$
			\begin{proof}
				Let $\bm{x}, \bm{y} \in \mathbb{R}^n, t \in \mathbb{R}$\\
					$\Rightarrow ||t \cdot \bm{y} + \bm{x}||^2 \geq 0$\\
					$\Rightarrow \sum\limits_{i=1}^n {(t \cdot y_i + x_i)}^2 \geq 0$\\
					$\Rightarrow \sum\limits_{i=1}^n (t^2 \cdot y_i^2 + 2x_i y_i t+x_i^2) \geq 0$\\
					$\Rightarrow t^2 \sum\limits_{i=1}^n y_i^2 + 2t \sum\limits_{i=1}^n x_i y_i + \sum\limits_{i=1}^n x_i^2 \geq 0$\\
					$\Rightarrow t^2 ||\bm{y}||^2 + 2t (\bm{x}\cdot\bm{y}) + ||\bm{x}||^2 \geq 0$\\
					$\Rightarrow a t^2+bt+c \geq 0$\\
					$\Rightarrow b^2-4ac \le 0$\\
					$\Rightarrow (2(\bm{x}\cdot\bm{y}))^2 - 4||\bm{y}||^2 ||\bm{x}||^2 \leq 0$\\
					$\Rightarrow 4(\bm{x}\cdot\bm{y})^2 - 4||\bm{y}||^2 ||\bm{x}||^2 \leq 0$\\
					$\Rightarrow (\bm{x}\cdot\bm{y})^2 - ||\bm{y}||^2 ||\bm{x}||^2 \leq 0$\\
					$\Rightarrow (\bm{x}\cdot\bm{y})^2 - (||\bm{x}|| \cdot ||\bm{y}||)^2 \leq 0$\\
					$\Rightarrow (\bm{x}\cdot\bm{y})^2 \leq (||\bm{x}|| \cdot ||\bm{y}||)^2$\\
					$\Rightarrow |\bm{x}\cdot\bm{y}| \leq ||\bm{x}|| \cdot ||\bm{y}||$
			\end{proof}
			\end{theorem}

			\begin{corollary}
				The Euclidean Norm is Subadditive.
			\end{corollary}
			\begin{proof}
				Let $\bm{x}, \bm{y} \in \mathbf{R}^n$. Then,
				\begin{align*}
					||\bm{x} + \bm{y}||^2 &= \sum\limits_{i=1}^n (x_i + y_i)^2\\
										  &= \sum\limits_{i=1}^n (x_i^2 + 2x_i y_i + y_i^2)\\
										  &= \sum\limits_{i=1}^n x_i^2 + 2 \sum\limits_{i=1}^n x_i y_i + \sum\limits_{i=1}^n y_i^2\\
										  &= ||\bm{x}||^2 + 2 (\bm{x} \cdot \bm{y}) + ||\bm{y}||^2\\
										  &\leq ||\bm{x}||^2 + 2 ||\bm{x}|| \cdot ||\bm{y}| + ||\bm{y}||^2 \tag{Cauchy Schwarz Inequality}\\
										  &= (||\bm{x}|| + ||\bm{y}||)^2\\
					  ||\bm{x} + \bm{y}|| &\leq ||\bm{x}|| + ||\bm{y}|| &&\qedhere
				\end{align*}
				
			\end{proof}

			\begin{theorem}[Euclidean Metric Space]
			\label{euclidean metric space}
			$(\mathbb{R}^n, d)$ forms a Metric Space where $d(\bm{x}, \bm{y}) = ||\bm{x} - \bm{y}||$.
			\end{theorem}
			\begin{proof}
				Let $\bm{x}, \bm{y} \in \mathbb{R}^n$ .
				\item \textbf{Symmetry:} $d(\bm{x},\bm{y}) = ||\bm{x}-\bm{y}|| = ||\bm{y}-\bm{x}|| = d(\bm{y},\bm{x})$
				\item \textbf{Identity of Indiscernibles:} $d(\bm{x},\bm{x}) = ||\bm{x}-\bm{x}|| = ||\bm{0}|| = 0$
				\item \textbf{Subadditivity:}
				\begin{align*}
					d(\bm{x},\bm{y}) &= ||\bm{x}-\bm{y}||\\
									 &= ||(\bm{x}-\bm{z}) + (\bm{z}-\bm{y})||\\
									 &\leq ||\bm{x}-\bm{z}|| + ||\bm{z}-\bm{y}|| \tag{Subadditivity of $||\cdot||$}\\
									 &= d(\bm{x},\bm{z}) + d(\bm{z},\bm{y}) &&\qedhere
				\end{align*}
			\end{proof}

			\begin{corollary}
				$(\mathbb{R}, d)$ forms a Metric Space where $d(x, y) = |x - y|$.
			\end{corollary}
			\begin{proof}
				Let n = 1. Then by Theorem \ref{euclidean metric space} we obtain a metric space
			\end{proof}

			\begin{theorem}[Taxicab Metric Space]
			\label{taxicab metric space}
				For any non-empty set $M$, the Taxicab Metric:
				\begin{equation*}
					d(x, y)=\begin{cases}
							  1 \quad & x \neq y \\
							  0 \quad & x = y \\
						 \end{cases}
				\end{equation*}
				forms a metric space $(M, d)$.
				\end{theorem}
				\begin{proof}
					Let $x, y, z \in E$.
					\item \textbf{Identity of Indiscernibles:} $d(x, x) = 0$.
					\item \textbf{Symmetry:} \\
						Let $x \neq y \Rightarrow d(x, y) = 1 = d(y, x)$.\\
						Let $x=y \Rightarrow d(x, y) = 0 = d(y, x)$.
					\item \textbf{Subadditivity:}
						\begin{equation*}
							d(x, z) + d(z, y)=\begin{cases}
									2 \quad & x \neq z \neq y \\
									0 \quad & x = y = z \\
									1 \quad &else
								\end{cases}
						\end{equation*}
						Therefore we have $d(x, y) \leq d(x, z) + d(z, y)$

				\end{proof}
		\subsection{Open and Closed Sets}
			Let $M$ be some Metric Space, $c \in M$, and $r \in \mathbf{R}^+$.
			\begin{definition}[Open Ball]
			\label{open ball}
				Then the Open Ball in $M$ of center $c$ and radius $r$ is the subset of $M$ given by
				$$B_r(c) = \{x \in M: d(x, c) < r \}$$
			\end{definition}
		
			\begin{definition}[Closed Ball]
			\label{closed ball}
				Then the Closed Ball in $M$ of center $c$ and radius $r$ is the subset of $M$ given by
				$$\bar{B}_r(c) = \{x \in M: d(x, c) \leq r \}$$
			\end{definition}

			\begin{theorem}
				Open/Closed Balls in $\mathbb{R}$ are Open/Closed Intervals.
			\end{theorem}
			\begin{proof}
				Let $c \in \mathbb{R}, r \in \mathbb{R}^+$.
				\begin{align*}
					B_r(c) &= \{ x \in M: d(x, c) < r \} \\
						   &= \{ x \in \mathbb{R}: |x - c| < r \} \\
						   &= \{ x \in \mathbb{R}: c - r < x < c + r \} \\
						   &= (c - r, c + r) \\
				\end{align*}
				\begin{align*}
					\bar{B}_r(c) &= \{ x \in M: d(x, c) \leq r \} \\
						   &= \{ x \in \mathbb{R}: |x - c| \leq r \} \\
						   &= \{ x \in \mathbb{R}: c - r \leq x \leq c + r \} \\
						   &= [c - r, c + r] \\
				\end{align*}
			\end{proof}

			\begin{definition}[Open Set]
				\label{open set}
				A subset S of a metric space M is open if every point in S is the center of some open ball contained in S:
				$$\forall c \in S, \exists r \in \mathbb{R}^+: B_{r}(c) \subset S $$
			\end{definition}

			\begin{theorem}
				In any metric space, an open ball is an open set.
			\end{theorem}
			\begin{proof} 
				Define $S = B_{\delta}(p)$ s.t.  $p \in M, \delta \in \mathbb{R}^+$.\\
				Let $c \in S$.\\
				$\Rightarrow d(c, p) < \delta$.\\
				Choose $r = \delta - d(c, p) \Rightarrow r \in \mathbb{R}^+$.\\
				Suppose $x \in B_{r}(c)$.\\
				$\Rightarrow d(x, c) < r$.\\
				$\Rightarrow d(x, c) < \delta - d(c, p)$.\\
				$\Rightarrow d(x, c) + d(c, p) < \delta$.\\
				$\Rightarrow d(x, p) < \delta$. (Triangle Inequality) \\
				$\Rightarrow x \in S$.\\
				$\Rightarrow B_{r}(c) \subset S$.
			\end{proof}

			\begin{theorem}
				For any metric space $(M, d)$:
				\begin{enumerate}
					\item $\varnothing$ is open
					\item M is open
					\item The union of an arbitrary number of open subsets is open: \\
					$\{V_{k}\}_{k \in \mathbb{N}}$ open in M $\Rightarrow \bigcup_{k \in \mathbb{N}} V_{k}$ open in M. 
					\item The intersection of a finite number of open subsets is open: \\
					$\{V_{k}\}_{k=1}^{n}$ open in M $\Rightarrow \bigcap_{k=1}^{n} V_{k}$ open in M.
				\end{enumerate}
			\end{theorem}
			\begin{proof} 
				Let $(M, d)$ be a metric space:
				\begin{enumerate}
					\item Let $c \in \varnothing$. But $c \notin \varnothing$. So trivially, $\exists r \in \mathbb{R^{+}}: B_{r}(c) \subset \varnothing$.
					\item Let $c \in M$. Suppose $x \in B_{r}(c) \Rightarrow d(x, c) < r \Rightarrow x \in M \Rightarrow B_{r}(c) \subset M$.
					\item Suppose $\{V_{k}\}_{k \in \mathbb{N}}$ arbitrary collection of open subsets in M.\\
						Let $c \in \bigcup_{k \in \mathbb{N}} V_{k}$.\\
						$\Rightarrow c \in V_{k}$ for some $k = 1, \ldots, n$.\\
						$\Rightarrow \exists r \in \mathbb{R^{+}}: B_{r}(c) \subset V_{k}$.\\
						$\Rightarrow \exists r \in \mathbb{R^{+}}: B_{r}(c) \subset \bigcup_{k \in \mathbb{N}} V_{k}$, since any set is contained in its union.
					\item Suppose $\{V_{k}\}_{k=1}^{n}$ finite collection of open subsets in M.\\
						Let $c \in \bigcap_{k=1}^{n} V_{k}$:
						$\Rightarrow c \in V_{k}, \forall k = 1, \ldots, n.$ \\
						$\Rightarrow \exists r_{k} \in \mathbb{R^{+}}: B_{r_{k}}(c) \subset V_{k}.$ \\
						Let $r = min(r_{1}, \ldots, r_{n})$. Suppose $x \in B_{r}(c)$:
						$\Rightarrow d(x, c) < r.$ \\
						$\Rightarrow d(x, c) < r_{1}, \ldots, d(x, c) < r_{n}.$ \\
						$\Rightarrow x \in B_{r_{1}}(c), \ldots, x \in B_{r_{n}}(c).$ \\
						$\Rightarrow x \in V_{1}, \ldots, x \in V_{n}.$ \\
						$\Rightarrow x \in \bigcap_{k=1}^{n} V_{k}.$ \\
						$\Rightarrow B_{r}(c) \subset \bigcap_{k=1}^{n} V_{k}.$
				\end{enumerate}
			\end{proof}

			\begin{definition}[Closed Set]
				\label{closed set}
				A subset $S$ of a metric space $(M, d)$ is closed if its complement $S^\complement$ is open in $(M, d)$.
			\end{definition}

			\begin{theorem}
				In any metric space, a closed ball is a closed set.
			\end{theorem}
			\begin{proof} 
				Define $S = \bar{B}_{\delta}(p)$ as some closed ball in $(M, d)$.\\
				Let $c \in S^\complement$.\\
				Then, $d(c, p) > \delta$.\\
				Choosing $r = d(c, p) - \delta$, clearly $r \in \mathbb{R}^+$.\\
				Suppose $x \in B_{r}(c)$.\\
				$\Rightarrow d(x, c) < r.$ \\
				$\Rightarrow d(x, c) < d(c, p) - \delta.$ \\
				$\Rightarrow d(c, p) - d(x, c) > \delta.$ \\
				$\Rightarrow d(c, x) + d(x, p) - d(x, c) > \delta.$ \\
				$\Rightarrow d(x, p) > \delta.$ \\
				$\Rightarrow x \in S^\complement.$ \\
				$\Rightarrow B_{r}(c) \subset S^\complement.$ \\
				$\Rightarrow S^\complement$ is open in $(M, d)$.\\
			\end{proof}

			\begin{theorem}
				For any metric space $(M, d)$:
				\begin{enumerate}
					\item $\varnothing$ is closed
					\item M is closed
					\item The intersection of an arbitrary number of closed subsets is closed: \\
					$\{V_{k}\}_{k \in \mathbb{N}}$ closed in $(M, d)$ $\Rightarrow \bigcap_{k \in \mathbb{N}} V_{k}$ closed in $(M, d)$ 
					\item The union of a finite number of closed subsets is closed: \\
					$\{V_{k}\}_{k=1}^{n}$ closed in $(M, d)$ $\Rightarrow \bigcup_{k=1}^{n} V_{k}$ closed in $(M, d)$
				\end{enumerate}
			\end{theorem}
			
			\begin{proof} 
				Let $(M, d)$ be a metric space:
				\begin{enumerate}
					\item $\varnothing^\complement = M \Rightarrow \varnothing^\complement$ open $\Rightarrow\varnothing$ closed.
					\item $M^\complement = \varnothing \Rightarrow M^\complement$ open $\Rightarrow M$ closed.
					\item Suppose $\{V_{k}\}_{k \in \mathbb{N}}$ arbitrary collection of closed subsets in $(M, d)$\\
					$\Rightarrow \{V_{k}^\complement\}_{k \in \mathbb{N}}$ is an abitrary collection of open subsets in $(M, d)$\\
					$\Rightarrow \bigcup_{k \in \mathbb{N}} V_{k}^\complement$ is open in $(M, d)$\\
					$\Rightarrow (\bigcap_{k \in \mathbb{N}} V_{k})^\complement$ is open in $(M, d)$\\
					$\Rightarrow \bigcap_{k \in \mathbb{N}} V_{k}$ is closed in $(M, d)$
					\item Suppose $\{V_{k}\}_{k=1}^{n}$ finite collection of closed subsets in $(M, d)$\\
					$\Rightarrow \{V_{k}^\complement\}_{k=1}^{n}$ is a finite collection of open subsets in $(M, d)$\\
					$\Rightarrow \bigcap_{k=1}^{n} V_{k}^\complement$ is open in $(M, d)$\\
					$\Rightarrow (\bigcup_{k=1}^{n} V_{k})^\complement$ is open in $(M, d)$\\
					$\Rightarrow \bigcup_{k=1}^{n} V_{k}$ is closed in $(M, d)$
				\end{enumerate}
			\end{proof}

			\begin{theorem}
				Any singleton set is closed.
			\end{theorem}
			\begin{proof}
				Define $S = \{p\}$ as the singleton set $\forall p \in (M, d)$.\\
				Let $c \in S^\complement = M / \{p\}$.\\
				Then, $c \neq p$.\\
				Choose $r = d(c, p) \Rightarrow r \in \mathbb{R}^+$.\\
				Suppose $x \in B_{r}(c)$.\\
				$\Rightarrow d(x, c) < r.$ \\
				$\Rightarrow d(x, c) < d(c, p).$ \\
				$\Rightarrow d(c, p) - d(x, c) > 0.$ \\
				$\Rightarrow d(p, c) - d(c, x) > 0.$ \\
				$\Rightarrow | d(p, c) - d(c, x) | > 0.$ \\
				$\Rightarrow d(p, x) > 0.$ \\
				$\Rightarrow p \neq x.$ \\
				$\Rightarrow x \in S^\complement.$ \\
				$\Rightarrow B_{r}(c) \subset S^\complement.$ \\
				So $S^\complement$ is open, making $S$ closed in $M$.
			\end{proof}

			\begin{corollary}
				Any finite subset of a metric space is closed.
			\end{corollary}
			\begin{proof}
				Let $x_1, \ldots, x_n \in M$.\\
				$\Rightarrow \{x_1\}, \ldots, \{x_n\}$ all closed in $M$.\\
				$\Rightarrow \{ x_1, \ldots, x_n \} = \bigcup_{k=1}^{n} \{x_k\}$ closed in $M$.\\
			\end{proof}

			\begin{theorem}
				Any sphere in a metric space is closed.
			\end{theorem}
			\begin{proof} 
				Let $c \in (M, d), r \in \mathbb{R}^+$.\\
				Define $r$-sphere centered at $c$ as $S = \{x \in (M, d): d(x, c) = r \}$.\\
				Let $V_1 = \bar{B}_{r}(c)$.\\
				$\Rightarrow V_1$ closed in $(M, d)$.\\
				Let $V_2 = B_{r}(c)^\complement$.\\
				$\Rightarrow V_2^\complement = B_{r}(c)$.\\
				$\Rightarrow V_2^\complement$ open in $(M, d)$.\\
				$\Rightarrow V_2$ closed in $(M, d)$.\\
				$\Rightarrow V_1 \cap V_2$ closed in $(M, d)$.\\
				$\Rightarrow \{x \in (M, d): d(x, c) \leq r \} \cap \{x \in (M, d): d(x, c) \geq r \}$ closed in $(M, d)$.\\
				$\Rightarrow \{x \in (M, d): d(x, c) = r \}$ closed in $(M, d)$.\\
				$\Rightarrow S$ closed in $(M, d)$.
			\end{proof}
			\begin{theorem}
				Any half-interval is neither open nor closed.
			\end{theorem}
			\begin{proof} 
				Let $a, b \in \mathbb{R}, a < b$.\\
				Define $S = [a, b)$ as a half-interval on $\mathbb{R}$.\\
				Let $r \in \mathbb{R}^+$.\\
				$\Rightarrow a - r < a$.\\
				$\Rightarrow a - r < min(S)$.\\
				$\Rightarrow a - r \in S^\complement$.\\
				$\Rightarrow \exists a \in S, \forall r \in \mathbb{R}^{+}: B_{r}(a) = (a - r, a + r) \subset S$.\\
				$\Rightarrow S$ is not open in $\mathbb{R}$.\\
				$\Rightarrow [b, \infty)$ is not open in $\mathbb{R}$ by same logic.\\
				$\Rightarrow S^\complement = (-\infty, a) \cup [b, \infty)$ is not open in $\mathbb{R}$ by same logic.\\
				$\Rightarrow S$ is not closed in $\mathbb{R}$.
			\end{proof}

			\begin{theorem}
				Any subspace of Euclidean space with a strictly bounded component is open. So, $\forall a \in \mathbb{R}, k \in \{ 1, \ldots, n \}$: 
				$\{ x \in \mathbb{R}^n: x_k < a \}$ and $\{ x \in \mathbb{R}^n: x_k > a \}$ are open in $\mathbb{R}^n$.
			\end{theorem}
			\begin{proof} 
				Let $a \in \mathbb{R}$, and $k \in \{ 1, \ldots, n \}$.\\
				Define $S = \{ x \in \mathbb{R}^n: x_k < a \}$ and $S' = \{ x \in \mathbb{R}^n: x_k > a \}$.\\
				Let $c \in S$ and $p \in S'$.\\
				$\Rightarrow c_k < a$ and $p_k > a$.\\
				Choose $r = a - c_k$ and $\delta = p_k - a$ $\Rightarrow$ $r, \delta \in \mathbb{R}^+$.\\
				Suppose $x \in B_{r}(c)$ and $y \in B_{\delta}(p)$.\\
				$\Rightarrow d(x, c) < r$ and $d(y, p) < \delta$.\\
				$\Rightarrow d(x, c) < a - c_k$ and $d(y, p) < p_k - a$.\\
				$\Rightarrow || x - c || < a - c_k$ and $|| y - p || < p_k - a$.\\
				$\Rightarrow \sqrt{\sum_{k} (x_k - c_k)^2} <  a - c_k$ and $\sqrt{\sum_{k} (y_k - p_k)^2} < p_k - a$.\\
				$\Rightarrow \sqrt{(x_k - c_k)^2} <  a - c_k$ and $\sqrt{(y_k - p_k)^2} < p_k - a$.\\
				$\Rightarrow | x_k - c_k | <  a - c_k$ and $| y_k - p_k | <  p_k - a$.\\
				$\Rightarrow x_k - c_k <  a - c_k$ and $p_k - y_k < p_k - a$.\\
				$\Rightarrow x_k <  a$ and $y_k >  a$.\\
				$\Rightarrow x \in S$ and $y \in S'$.\\
				$\Rightarrow B_{r}(c) \subset S$ and $B_{\delta}(p) \subset S'$.
			\end{proof}

			\begin{theorem}
				Any subspace of Euclidean space with a weakly bounded component is closed. So, $\forall a \in \mathbb{R}, k \in \{ 1, \ldots, n \}$: 
				$\{ x \in \mathbb{R}^n: x_k \leq a \}$ and $\{ x \in \mathbb{R}^n: x_k \geq a \}$ are closed in $\mathbb{R}^n$.
			\end{theorem}

			\begin{proof} 
				Let $a \in \mathbb{R}$, and $k \in \{ 1, \ldots, n \}$.\\
				Define $S = \{ x \in \mathbb{R}^n: x_k \leq a \}$ and $S' = \{ x \in \mathbb{R}^n: x_k \geq a \}$.\\
				$\Rightarrow S^\complement = \{ x \in \mathbb{R}^n: x_k > a \}$ and $S'^\complement = \{ x \in \mathbb{R}^n: x_k < a \}$.\\
				$\Rightarrow S^\complement, S'^\complement$ are open in $\mathbb{R}^n$.\\
				$\Rightarrow S, S'$ are closed in  $\mathbb{R}^n$.
			\end{proof}
			
			\begin{definition}[Open/Closed Intervals in $\mathbb{R}^n$]
				\label{open closed intervals}
				Let $a_1, \ldots, a_n, b_1, \ldots b_n \in \mathbb{R}$.
				For $a_1 < b_1, \ldots, a_n < b_n$, the open interval in $\mathbb{R}^n$ is:
				$$ \{ x \in \mathbb{R}^n: a_k < x_k < b_k, k = 1, \ldots, n \}$$
				For $a_1 \leq b_1, \ldots, a_n \leq b_n$, the closed interval in $\mathbb{R}^n$ is:
				$$ \{ x \in \mathbb{R}^n: a_k \leq x_k \leq b_k, k = 1, \ldots, n \}$$
			\end{definition}

			\begin{theorem}
				Any open/closed interval in Euclidean space is open/closed set.
			\end{theorem}
			
			\begin{proof} 
				Let $S = \{ x \in \mathbb{R}^n: a_k < x_k < b_k, k = 1, \ldots, n \}$\\
				$\Rightarrow S = \bigcap_{k=1}^{n} \{ x \in \mathbb{R}^n: a_k < x_k < b_k \}$.\\
				$\Rightarrow S = (\bigcap_{k=1}^{n} \{ x \in \mathbb{R}^n: x_k > a_k \}) \cap (\bigcap_{k=1}^{n} \{ x \in \mathbb{R}^n: x_k < b_k \})$.\\
				$\Rightarrow S$ is open, since it is the finite intersection of ($2n$) open sets.\\
				Define $S' = \{ x \in \mathbb{R}^n: a_k \leq x_k \leq b_k, k = 1, \ldots, n \}$.\\
				$\Rightarrow S' = \bigcap_{k=1}^{n} \{ x \in \mathbb{R}^n: a_k \leq x_k \leq b_k \}$.\\
				$\Rightarrow S' = (\bigcap_{k=1}^{n} \{ x \in \mathbb{R}^n: x_k \geq a_k \}) \cap (\bigcap_{k=1}^{n} \{ x \in \mathbb{R}^n: x_k \leq b_k \})$.\\
				$\Rightarrow S'$ is closed, since it is the finite intersection of ($2n$) closed sets.
			\end{proof}

			\begin{definition}[Bounded]
				\label{bounded}
				A subset $S$ of a metric space $M$ is bounded if it is contained in some ball.
			\end{definition}

			\begin{theorem}
				Any open/closed interval in Euclidean space is bounded.
			\end{theorem}
			\begin{proof}
				Let $a, b \in \mathbb{R}^n$ s.t. $a \neq b$\\
				Define $S = \{ x \in \mathbb{R}^n: a_k \leq x_k \leq b_k, k = 1, \ldots, n \}$\\
				Choose $r = d(b, a) > 0$\\
				Let $x \in S$\\
				$\Rightarrow 0 \leq x_k - a_k \leq b_k - a_k$\\
				$\Rightarrow (x_k - a_k)^2 \leq (b_k - a_k)^2$\\
				$\Rightarrow \sum \limits_{k=1}^n (x_k - a_k)^2 \leq \sum \limits_{k=1}^n (b_k - a_k)^2$\\
				$\Rightarrow \sqrt{\sum \limits_{k=1}^n (x_k - a_k)^2} \leq \sqrt{\sum \limits_{k=1}^n (b_k - a_k)^2}$\\
				$\Rightarrow ||x - a|| \leq ||b - a||$\\
				$\Rightarrow d(x, a) \leq d(b, a)$\\
				$\Rightarrow d(x, a) \leq r$\\
				$\Rightarrow x \in \bar{B}_r(a)$\\
				$\Rightarrow S \subset \bar{B}_r(a)$\\
				Let $S' = \{ x \in \mathbb{R}^n: a_k < x_k < b_k, k = 1, \ldots, n \}$\\
				$\Rightarrow S' \subset S$\\
				$\Rightarrow S' \subset \bar{B}_r(a)$
			\end{proof}

			\begin{theorem}
				The union of a finite collection of bounded subsets of a metric space is bounded.
			\end{theorem}
			\begin{proof}
				Suppose $V_1, \ldots, V_n$ are bounded subsets in $M$.\\
				Let $c_1 \in V_1, \ldots, c_n \in V_n$.\\
				$\Rightarrow \exists r_1, \ldots, r_n \in \mathbb{R}^+$ s.t. $V_{k} \subset \bar{B}_{r_k}(c_k), \forall k = 1, \ldots, n$.\\
				Let $x \in \bigcup_{k=1}^{n} V_k$.\\
				$\Rightarrow x \in V_k$ for some $k = 1, \ldots, n$.\\
				$\Rightarrow x \in \bar{B}_{r_k}(c_k)$.\\
				$\Rightarrow \bigcup_{k=1}^{n} V_k \subset  \bar{B}_{r_k}(c_k)$.
			\end{proof}

			\begin{theorem}
				A nonempty closed subset of $\mathbb{R}$, if it is bounded from above has a greatest element
				and if it is bounded from below has a least element.
			\end{theorem}
			\begin{proof}
				Let $S \subset \mathbb{R}, S \neq \varnothing$.\\
				Suppose S is closed in $\mathbb{R}$ and bounded above.\\
				Let $c = sup(S)$.\\
				With an eye to contradict, assume $c \in S^\complement$.\\
				Since $S^\complement$ open in $\mathbb{R}$, $\exists r \in \mathbb{R}^+$ s.t. $(c-r, c+r) \subset S^\complement$.\\
				Then no element in S is greater than $c - r$.\\
				$\Rightarrow c - r$ is an upper bound for $S$.\\
				This must be a contradictions, so $c \in S$.
			\end{proof}

		\subsection{Convergent Sequences}

			\begin{definition}[Convergent Sequence]
				\label{convergent sequence}
				Let $\{ p_n \}_{n \in \mathbb{N}}$
				
			\end{definition}
		
		\subsection{Completeness}
		
		\subsection{Compactness}
		
		\subsection{Connectedness}
	
	\section{Continuous Functions}
		\subsection{Definition of Continuity. Examples}
		
		\subsection{Continuity and Limits}
		
		\subsection{The Continuity of Rational Operations. Functions with values in $E^n$}
		
		\subsection{Continuous Functions on a Compact Metric Space}
		
		\subsection{Continuous Functions on a Connected Metric Space}
		
		\subsection{Sequences of Functions}
	
	\section{Differentiation}
		\subsection{Definition of the Derivative}
		
		\subsection{Rules of Differentiation}		
		
		\subsection{The Mean Value Theorem}
		
		\subsection{Taylor's Theorem}	
	
	\section{Riemann Integration}
		\subsection{Definition and Examples}
		
		\subsection{Linearity and Order Properties of the Integral}
		
		\subsection{Existence of the Integral}
		
		\subsection{The Fundamental Theorem of Calculus}
		
		\subsection{The Logarithmic and Exponential Functions}
		
		\subsection{Definition of Continuity. Examples}
	
	
	\section{Interchange of Limit Operations}
		\subsection{Integration and Differentiation of Sequences of Functions}
		
		\subsection{Infinite Series}
		
		\subsection{Power Series}
		
		\subsection{The Trigonometric Functions}
		
		\subsection{Differentiation under the Integral Sign}
		
		
	\section{The Method of Successive Approximations}
		\subsection{The Fixed Point Theorem}
		
		\subsection{The Simplest Case of the Implicit Function Theorem}
		
		\subsection{Existence and Uniqueness Theorems for Ordinary Differential Equations}
	
	
	\section{Partial Differentiation}
		\subsection{Definitions and Basic Properties}
		
		\subsection{Higher Derivatives}
		
		\subsection{The Implicit Function Theorem}
	
	
	\section{Multiple Integrals}
		\subsection{Riemann Integration on Closed Intervals of $E^n$. Examples and Basic Properties}
		
		\subsection{Existence of the Integral. Integration on Arbitrary Subsets of $E^n$. Volume}
		
		\subsection{Iterated Integrals}
		
		\subsection{Change of Variable}


	
\end{document}