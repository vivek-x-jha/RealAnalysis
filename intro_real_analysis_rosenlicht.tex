\documentclass{article}
\usepackage[utf8]{inputenc}

\title{Intro Real Analysis}
\author{Rosenlicht}
\date{April 2022}

\usepackage{amsmath}
\usepackage{amsthm}
\usepackage{amssymb}

\theoremstyle{definition}
\newtheorem{definition}{Definition}[section]

\theoremstyle{theorem}
\newtheorem{theorem}{Theorem}[section]

\begin{document}
	
	\maketitle
	\tableofcontents
	
	\section{Notions from Set Theory}
		\subsection{Sets and Elements. Subsets}
		
		\subsection{Operations on Sets}
		
		\subsection{Functions}
		
		\subsection{Finite and Infinite Sets}
	
	\section{The Real Number System}
		\subsection{The Field Properties}
		
		\subsection{Order}
		
		\subsection{The Least Upper Bound Property}
		
		\subsection{The existence of Square Roots}

	
	\section{Metric Spaces}
		\subsection{Definition of Metric Spaces. Examples}
			
			\begin{definition}[Metric Space]
				A Metric Space is an ordered pair $(E, d)$ where E is some set along with a metric function $d \colon E \times E \to \mathbb{R}$
				 such that:
				 \begin{enumerate}
				 	\item \textbf{Identity of Indiscernibles:} $d(p,q) = 0 \Leftrightarrow p=q, \forall p,q \in E$
				 	\item \textbf{Symmetry:} $d(p,q) = d(q,p), \forall p,q \in E$
				 	\item \textbf{Subadditivity:} $d(p,q) \leq d(p,r) + d(r,q), \forall p,q,r \in E$
				 \end{enumerate}
			\end{definition}
		
			\begin{theorem}
				Metric functions are non-negative
			\end{theorem}
			\begin{proof}
				Let d be a a metric function with points p,q $\in$ E. So,\\
				\begin{align*}
					d(p,p) &\leq d(p,q) + d(q,p) \tag{Subadditivity}. \\
					d(p,p) &\leq d(p,q) + d(p,q) \tag{Symmetry}. \\
					d(p,p) &\leq 2 \cdot d(p,q). \\
					0 &\leq 2 \cdot d(p,q) \tag{Identity of Indiscernibles}. \\
					0 &\leq d(p,q). &&\qedhere
				\end{align*}
			\end{proof}

		

		\subsection{Open and Closed Sets}
		
		\subsection{Convergent Sequences}
		
		\subsection{Completeness}
		
		\subsection{Compactness}
		
		\subsection{Connectedness}
	
	\section{Continuous Functions}
		\subsection{Definition of Continuity. Examples}
		
		\subsection{Continuity and Limits}
		
		\subsection{The Continuity of Rational Operations. Functions with values in $E^n$}
		
		\subsection{Continuous Functions on a Compact Metric Space}
		
		\subsection{Continuous Functions on a Connected Metric Space}
		
		\subsection{Sequences of Functions}
	
	\section{Differentiation}
		\subsection{Definition of the Derivative}
		
		\subsection{Rules of Differentiation}		
		
		\subsection{The Mean Value Theorem}
		
		\subsection{Taylor's Theorem}	
	
	\section{Riemann Integration}
		\subsection{Definition and Examples}
		
		\subsection{Linearity and Order Properties of the Integral}
		
		\subsection{Existence of the Integral}
		
		\subsection{The Fundamental Theorem of Calculus}
		
		\subsection{The Logarithmic and Exponential Functions}
		
		\subsection{Definition of Continuity. Examples}
	
	
	\section{Interchange of Limit Operations}
		\subsection{Integration and Differentiation of Sequences of Functions}
		
		\subsection{Infinite Series}
		
		\subsection{Power Series}
		
		\subsection{The Trigonometric Functions}
		
		\subsection{Differentiation under the Integral Sign}
		
		
	\section{The Method of Successive Approximations}
		\subsection{The Fixed Point Theorem}
		
		\subsection{The Simplest Case of the Implicit Function Theorem}
		
		\subsection{Existence and Uniqueness Theorems for Ordinary Differential Equations}
	
	
	\section{Partial Differentiation}
		\subsection{Definitions and Basic Properties}
		
		\subsection{Higher Derivatives}
		
		\subsection{The Implicit Function Theorem}
	
	
	\section{Multiple Integrals}
		\subsection{Riemann Integration on Closed Intervals of $E^n$. Examples and Basic Properties}
		
		\subsection{Existence of the Integral. Integration on Arbitrary Subsets of $E^n$. Volume}
		
		\subsection{Iterated Integrals}
		
		\subsection{Change of Variable}

	
\end{document}
